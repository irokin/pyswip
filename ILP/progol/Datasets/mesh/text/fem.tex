\documentstyle[12pt,a4]{article}
\input{epsf}

\begin{document}
\section*{Learning rules for finite element mesh design}
Finite element methods are used extensively by engineers and
modelling scientists to analyse stresses in physical
structures.  These structures are represented quantitatively
as finite collections of elements. The deformation of
each element is computed using linear algebraic
equations. In order to design a numerical
model of a physical structure it is necessary to decide
the appropriate resolution for modelling each
component part. Considerable expertise is
required in choosing these resolution
values. Too fine a mesh leads to unnecessary
computational overheads when executing the model.
Too coarse a mesh produces intolerable approximation errors.

% Here is an example of an object that has been divided into
% meshes that are too fine (left) and too coarse (middle) and
% just right (right).
% \begin{figure*}[htb]
% \vspace{6cm}
% \end{figure*}
% 

The resolution of a FE mesh is determined by the number of
elements on each of its edges. The problem of learning rules for
determining the resolution of a FE mesh is therefore,
to learn rules that determine the number of
elements on an edge. The data here is from 
experiments conducted with {\em Golem\/} \cite{mugfeng:golem}
(reported in \cite{dolmugg:fem}). The task is to learn
rules for the number of elements using the following
information:

\begin{itemize}
\item The training examples have the form
	{\bf mesh(Edge,Number\_of\_elements)}, where {\bf Edge} is an edge label
	(unique for each edge) and {\bf Number\_of\_elements} is the number
	of elements on the edge denoted by label {\bf Edge}. The number of elements
	on an edge varies from 1 to 17.
\item The background knowledge describes some of the factors that
	influence the resolution of a FE mesh, such as the type of edges,
	boundary conditions and loadings, as well as the shape of the
	structure (relations of neighborhood etc.).
\item According to its importance and geometric shape, an edge can belong to one
	of the following types: {\bf important\_long}, {\bf important},
	{\bf important\_short}, {\bf not\_important}, {\bf circuit},
	{\bf half\_circuit}, {\bf quarter\_circuit}, {\bf short\_for\_hole},
	{\bf long\_for\_hole}, {\bf circuit\_hole}, {\bf half\_circuit\_hole} and
	{\bf quarter\_circuit\_hole}. 
\item With respect to the boundary conditions an edge can be {\bf free},
	{\bf one\_side\_fixed}, {\bf two\_side\_fixed} or {\bf fixed}.
\item According to the loadings an edge is {\bf not\_load\-ed},
	{\bf one\_side\_load\-ed}, {\bf two\_si\-de\_load\-ed}  or
	{\bf continuously\_load\-ed}.
\item Background knowledge about the shape of a structure includes the
	symmetric relations {\bf neighbour/2} and {\bf opposite/2}, as well as the
	relation {\bf equal/2}.
\end{itemize}

The data concerns five structures labelled ``a'' -- ``e''. For each
structure, data files are as used in the original {\em Golem\/}
experiments. That is, background knowledge files have a ``.b'' suffix,
positive example files have a ``.f'' suffix, and negative example files
have a ``.n'' suffix. The files are all stored in one compressed TAR file.

\begin{thebibliography}{}

\bibitem{dolmugg:fem}
Dolsak B. and Muggleton S. (1992).
The application of Inductive Logic Programming to 
finite element mesh design.
In S. Muggleton editor.
{\sl Inductive Logic Programming}, Academic Press, London.

\bibitem{mugfeng:golem}
Muggleton S.H. and Feng C. (1990).
Efficient induction of logic programs.
{\sl Proceedings of the First Conference on Algorithmic Learning Theory},
Tokyo.


\end{thebibliography}

\end{document}
