\documentstyle[11pt]{article}
\begin{document}
\input{epsf}
\title{On the rails}
\author{Donald Michie}
\date{5th July, 1994.}
\maketitle

The diagram below originated 20 years ago from Ryszard Michalski, 
pioneer of the branch of machine learning that digs theories out of 
data. 

\centerline{\epsfxsize=\textwidth\epsfbox{michalski.ps}}

As in scientific discovery, it is required to conjecture some 
plausible Law, in this case  governing what kinds of trains are 
Eastbound and what kinds Westbound. 

The newly discovered Law can be tested on fresh observations. If it 
predicts correctly, it is said to receive corroboration.  Otherwise it 
is scrapped, or alternatively patched up. After Galileo's 
introduction of the  telescope the Ptolemaic theory of the heavens was
eventually scrapped, and replaced by one in which the earth goes 
round the sun. 

I published the trains diagram in the  computer press over ten years 
ago. My post-bag contained some neat conjectures from readers, such as: 
\begin{description}
\item[Theory A:] If a train has a short closed car, then it is Eastbound and 
otherwise Westbound. 

\item[Theory B:] If a train has two cars, or has a car with a corrugated roof, 
then it is Westbound and otherwise Eastbound. 

One and the same set of observations can support several different 
theories. Pending new observations we generally take the simplest and 
hope for the best. Theory A is marginally simpler than B and a good 
deal simpler than C from one of my readers: 

\item[Theory C:] If a train has more than two different kinds of load, then it 
is Eastbound and otherwise Westbound. 

No learning system of those days was capable of coming up with a theory
 like C, still less one like the following from another reader: 
 
\item[Theory D:] For each train add up the total number of sides of loads 
(taking a circle to have one side). If the answer is a divisor of 60 
then the train is Westbound and otherwise Eastbound. 
\end{description}
Time has moved on and I have meanwhile observed ten more trains:

\centerline{\epsfxsize=\textwidth\epsfbox{new.ps}}

Merging these with Michalski's original ten, we note that Theories 
A, B, C and D all fail. Can the mental ingenuity of Computing 
readers extract sense from the enlarged sample and give us a new Law, 
fitting all 20?  The best entry, judged on accuracy and simplicity, wins 
a free copy of Richard Gregory's handsome book The Oxford Companion to 
the Mind, by kind donation of the Oxford University Press. I will also 
publish any good Laws of machine authorship.  To be in line for a prize 
the programmers must submit their names and addresses, as must the other 
entrants. 
\end{document}
